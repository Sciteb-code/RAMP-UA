\documentclass{article}
\usepackage[utf8]{inputenc}

\usepackage{draftwatermark} % To make a watermark
%\SetWatermarkText{Confidential} % COnfigure the watermark 
%\SetWatermarkScale{5}

\title{RAMP Dynamic Model}
\author{Nick Malleson}
\date{}

\usepackage{natbib}
\usepackage{graphicx}

\begin{document}

\maketitle
%\begin{abstract}
%    This will be the abstract
%\end{abstract}
\tableofcontents

%\newpage

\section{Introduction}

This document will introduce and explain how the dynamic model works.

\section{Stage 1: Individuals deliver risk to locations}

In each iteration the synthetic individuals visit some locations (shops, schools, etc.).
If they are infected then they impart some of this infection risk on to the location.

\textit{Here is one way that we might calculate the `danger' associated with each location. I've no idea if this is a good way to do it, we need to discuss. Also I suspect that the language used here is not appropriate and will confuse people from disciplines where these words have precise meanings.}

The danger, $D$, associated with a location, $l$ is calculated by summing the individual risks, $r$, imparted by each agent/individual, $a$, from a total population of $N$ agents, as they visit the location:
\begin{equation}
  D_l = \sum_{a=0}^{N}  a_r  
\end{equation}
and where the individual risk is a made up of the proportion of time per day that the individual spends doing that activity, $t$, and the proportion of visits that the individual makes to that particular location, $p$\footnote{Individuals may visit many different locations, so for each hour they spend doing a particular activity that time is distributed among the possible locations that they might visit.}:
\begin{equation}
  a_r(l) =  t  p  
\end{equation}
Also assume that if an infected individual spends 24 hours per day in a location, then they will impart a `danger' of 1 on to that location. 

\textit{Is that enough to put some numbers on the `danger' value? Will obviously need to calibrate the later stages carefully.}

\section{Stage 2: Individuals receive risks from locations}

In the second stage of each iteration, individuals receive some `exposure' as they visit locations.

\textit{Presumably this will vary by the disease status of the individual. Lets assume that the following applies to a susceptible individual, not one who is pre- or a-symptomatic or recovered}

The exposure, $e$, that an individual receives per day, is the summation of the danger, $D$ of all the locations that they visit, $L$, proportioned by the amount of time they spend there and the proportion of visits to that particular location that they make:
\begin{equation}
  a_e = \sum_{l=0}^{L} D_l t_l p_l 
\end{equation}

\section{Worked Example}

\textit{Not exactly sure there is much value in explaining further, but a small simulation but be instructive..}



\newpage
(Ignore this, I need to include one reference to get bibtex working: \cite{hellewell_feasibility_2020})

\bibliographystyle{plain}
\bibliography{ramp-refs}

\end{document}

